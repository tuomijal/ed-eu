\documentclass{article}
\usepackage{booktabs}
\usepackage{graphicx}
\usepackage{lipsum}
\usepackage{float}
\usepackage{subcaption}
\usepackage{multirow}
\usepackage{rotating}
\usepackage{makecell}
\usepackage{xcolor}
\usepackage{amsmath}


\title{European Emergency Department Crowding Study}
\author{EEDCS consortium}

\begin{document}
\maketitle


\begin{abstract}
    \lipsum[1]    
\end{abstract}

\section{Introduction}

Emergency department is both a door and a window. For the patient it is a door to sometimes lifesaving care and always to instant care when all other healthcare system constituents have failed. For a healthcare operations researcher it is a window to 

Healthcare systems in developed countries are struggling with the burden caused by ageing populations. One of the most prominent manifestations of this mismatch between supply and demand is 

\textcolor{red}{Lisäanalyysi: ruuhka terveydenhuollon kustannusten funktiona}
\textcolor{red}{Lisäanalyysi: ruuhka yksityinen vs. julkinen terveydenhuolto}

Every year an increasing number of forecasting articles are published but no one knows what is the most important target to forecast. Association between crowding and mortality has been repeatedly documented but no one knows what is the definition for crowding that is actually used in practice.

In this article we aim to address these issues by engaging with the emergency department physicians and asking these questions directly from them. 

\section{Materials and methods}
\lipsum[9-11]

Respondent level: Unprocessed respondent data

Hospital-level mean

\begin{equation}
\bar{y}_{jk} = \frac{1}{n_{jk}} \sum_{i=1}^{n_{jk}} y_{ijk}
\end{equation}

where $y_{ijk}$ is response of respondent $i$ in hospital j in country k,
and $n_{jk}$ is number of respondents in that hospital.

Country-level mean (mean of hospital means)

\begin{equation}
\bar{y}{k}^{(H)} = \frac{1}{m_k} \sum{j=1}^{m_k} \bar{y}_{jk}
\end{equation}

where $m_k = number of hospitals in country k$.

(If you prefer the respondent-weighted version instead, use)

\begin{equation}
\bar{y}{k}^{(R)} = \frac{1}{N_k} \sum{j=1}^{m_k} n_{jk} \, \bar{y}_{jk}
\end{equation}

with \[N_k = \sum_{j=1}^{m_k} n_{jk}.\]

Europe-level population-weighted mean

\begin{equation}
\bar{y}{E} = \frac{\sum{k=1}^{K} w_k \, \bar{y}{k}^{(H)}}{\sum{k=1}^{K} w_k}
\end{equation}

where $w_k$ is population weight (or sample weight) for country k,
and K is number of countries in Europe.



\section{Results}


\subsection{Respondents}
\lipsum[4-6]
% \begin{table}[h]
\caption{Repondent positions}
\label{tab:respondents}
\begin{tabular}{lrrrrr}
\toprule
 & Head & Other & Resident & Specialist & Trainee \\
country &  &  &  &  &  \\
\midrule
Albania & 3 & 0 & 0 & 4 & 0 \\
Austria & 1 & 0 & 0 & 3 & 0 \\
Belgium & 1 & 0 & 0 & 0 & 0 \\
Croatia & 4 & 0 & 4 & 8 & 0 \\
Denmark & 1 & 0 & 3 & 10 & 0 \\
Estonia & 3 & 1 & 0 & 5 & 0 \\
Finland & 7 & 1 & 3 & 3 & 0 \\
France & 3 & 0 & 1 & 11 & 0 \\
Germany & 20 & 3 & 0 & 5 & 1 \\
Greece & 4 & 0 & 3 & 5 & 0 \\
Hungary & 7 & 0 & 1 & 3 & 0 \\
Iceland & 1 & 0 & 0 & 4 & 0 \\
Ireland & 1 & 0 & 1 & 5 & 1 \\
Italy & 8 & 3 & 0 & 9 & 0 \\
Lithuania & 0 & 0 & 0 & 1 & 0 \\
Malta & 0 & 0 & 0 & 7 & 1 \\
Netherlands & 0 & 0 & 0 & 6 & 0 \\
Norway & 3 & 0 & 0 & 1 & 0 \\
Poland & 8 & 1 & 3 & 4 & 0 \\
Romania & 3 & 0 & 0 & 2 & 0 \\
Slovenia & 2 & 0 & 1 & 1 & 0 \\
Spain & 3 & 1 & 0 & 9 & 0 \\
Sweden & 2 & 1 & 0 & 7 & 0 \\
Switzerland & 1 & 0 & 0 & 0 & 0 \\
Turkey & 0 & 6 & 16 & 14 & 3 \\
United Kingdom & 5 & 14 & 29 & 36 & 8 \\
\bottomrule
\end{tabular}
\end{table}

% \begin{table}[h]
\caption{Hospital types}
\label{tab:hospitals}
\begin{tabular}{lrrrrr}
\toprule
 & other & primary & secondary & specialized & tertiary \\
country &  &  &  &  &  \\
\midrule
Albania & 0 & 1 & 3 & 0 & 3 \\
Austria & 0 & 0 & 0 & 0 & 4 \\
Belgium & 0 & 0 & 0 & 0 & 1 \\
Croatia & 0 & 2 & 1 & 0 & 13 \\
Denmark & 0 & 0 & 8 & 0 & 6 \\
Estonia & 0 & 2 & 2 & 0 & 5 \\
Finland & 0 & 0 & 8 & 0 & 6 \\
France & 0 & 3 & 1 & 1 & 10 \\
Germany & 0 & 0 & 8 & 0 & 21 \\
Greece & 0 & 0 & 2 & 0 & 10 \\
Hungary & 1 & 0 & 5 & 0 & 5 \\
Iceland & 0 & 1 & 0 & 0 & 4 \\
Ireland & 1 & 0 & 2 & 0 & 5 \\
Italy & 0 & 3 & 4 & 2 & 11 \\
Lithuania & 0 & 0 & 0 & 0 & 1 \\
Malta & 0 & 1 & 1 & 0 & 6 \\
Netherlands & 0 & 0 & 4 & 0 & 2 \\
Norway & 0 & 0 & 3 & 0 & 1 \\
Poland & 1 & 0 & 6 & 0 & 9 \\
Romania & 0 & 0 & 1 & 0 & 4 \\
Slovenia & 0 & 0 & 1 & 0 & 3 \\
Spain & 0 & 3 & 2 & 1 & 7 \\
Sweden & 1 & 2 & 2 & 0 & 5 \\
Switzerland & 0 & 0 & 0 & 0 & 1 \\
Turkey & 1 & 0 & 4 & 2 & 32 \\
United Kingdom & 2 & 5 & 49 & 3 & 33 \\
\bottomrule
\end{tabular}
\end{table}

\begin{table}[H]
\caption{Description}
\label{tab:description}
\footnotesize
\makebox[\textwidth][c]{%
\resizebox{1.2\textwidth}{!}{%
\begin{tabular}{p{1.0cm}p{1.1cm}p{0.07cm}p{0.07cm}p{0.07cm}p{0.07cm}p{0.07cm}p{0.07cm}p{0.07cm}p{0.07cm}p{0.07cm}p{0.07cm}p{0.07cm}p{0.07cm}p{0.07cm}p{0.07cm}p{0.07cm}p{0.07cm}p{0.07cm}p{0.07cm}p{0.07cm}p{0.07cm}p{0.07cm}p{0.07cm}p{0.07cm}p{0.07cm}p{0.07cm}p{0.07cm}p{0.07cm}}
\toprule
 &  & \rotatebox{90}{Albania} & \rotatebox{90}{Austria} & \rotatebox{90}{Belgium} & \rotatebox{90}{Croatia} & \rotatebox{90}{Denmark} & \rotatebox{90}{Estonia} & \rotatebox{90}{Finland} & \rotatebox{90}{France} & \rotatebox{90}{Germany} & \rotatebox{90}{Greece} & \rotatebox{90}{Hungary} & \rotatebox{90}{Iceland} & \rotatebox{90}{Ireland} & \rotatebox{90}{Italy} & \rotatebox{90}{Lithuania} & \rotatebox{90}{Malta} & \rotatebox{90}{Netherlands} & \rotatebox{90}{Norway} & \rotatebox{90}{Poland} & \rotatebox{90}{Romania} & \rotatebox{90}{Slovenia} & \rotatebox{90}{Spain} & \rotatebox{90}{Sweden} & \rotatebox{90}{Switzerland} & \rotatebox{90}{Turkey} & \rotatebox{90}{United Kingdom} & \rotatebox{90}{Total} \\
\midrule
\multirow[t]{5}{*}{Position} & Other & 0 & 0 & 0 & 0 & 0 & 1 & 1 & 0 & 3 & 0 & 0 & 0 & 0 & 3 & 0 & 0 & 0 & 0 & 1 & 0 & 0 & 1 & 1 & 0 & 6 & 14 & 31 \\
 & Trainee & 0 & 0 & 0 & 0 & 0 & 0 & 0 & 0 & 1 & 0 & 0 & 0 & 1 & 0 & 0 & 1 & 0 & 0 & 0 & 0 & 0 & 0 & 0 & 0 & 3 & 8 & 14 \\
 & Resident & 0 & 0 & 0 & 4 & 3 & 0 & 3 & 1 & 0 & 3 & 1 & 0 & 1 & 0 & 0 & 0 & 0 & 0 & 3 & 0 & 1 & 0 & 0 & 0 & 16 & 29 & 65 \\
 & Specialist & 4 & 3 & 0 & 8 & 10 & 5 & 3 & 11 & 5 & 5 & 3 & 4 & 5 & 9 & 1 & 7 & 6 & 1 & 4 & 2 & 1 & 9 & 7 & 0 & 14 & 36 & 163 \\
 & Head & 3 & 1 & 1 & 4 & 1 & 3 & 7 & 3 & 20 & 4 & 7 & 1 & 1 & 8 & 0 & 0 & 0 & 3 & 8 & 3 & 2 & 3 & 2 & 1 & 0 & 5 & 91 \\
\cline{1-29}
\multirow[t]{4}{*}{Hospital} & Other & 0 & 0 & 0 & 0 & 0 & 0 & 0 & 0 & 0 & 0 & 1 & 0 & 1 & 0 & 0 & 0 & 0 & 0 & 1 & 0 & 0 & 0 & 1 & 0 & 1 & 2 & 7 \\
 & Primary & 1 & 0 & 0 & 2 & 0 & 2 & 0 & 3 & 0 & 0 & 0 & 1 & 0 & 3 & 0 & 1 & 0 & 0 & 0 & 0 & 0 & 3 & 2 & 0 & 0 & 5 & 23 \\
 & Secondary & 3 & 0 & 0 & 1 & 8 & 2 & 8 & 1 & 8 & 2 & 5 & 0 & 2 & 4 & 0 & 1 & 4 & 3 & 6 & 1 & 1 & 2 & 2 & 0 & 4 & 49 & 117 \\
 & Tertiary & 3 & 4 & 1 & 13 & 6 & 5 & 6 & 11 & 21 & 10 & 5 & 4 & 5 & 13 & 1 & 6 & 2 & 1 & 9 & 4 & 3 & 8 & 5 & 1 & 34 & 36 & 217 \\
\cline{1-29}
 & Total & 7 & 4 & 1 & 16 & 14 & 9 & 14 & 15 & 29 & 12 & 11 & 5 & 8 & 20 & 1 & 8 & 6 & 4 & 16 & 5 & 4 & 13 & 10 & 1 & 39 & 92 & 364 \\
\cline{1-29}
\bottomrule
\end{tabular}
}}
\end{table}


% \begin{figure}[H]
%     \centering
%         \includegraphics[width=0.7\textwidth]{../output/plots/country_freqs}
%         \caption{countryfreqs}
%         \label{fig:country_freqs}
% \end{figure}

\subsection{Causes}
\lipsum[4-6]

\begin{figure}[p]
    \centering
        \includegraphics[height=1.0\textheight]{../output/plots/causes}
        \caption{Perceived causes of crowding based on EU-level statistics.}
        \label{fig:causes}
\end{figure}

\subsection{Operations management}
\lipsum[4]

\begin{figure}[H]
    \centering
        \includegraphics[width=1.0\textwidth]{../output/plots/severity}
        \caption{severity}
        \label{fig:severity}
\end{figure}

\lipsum[5]

\begin{figure}[H]
    \centering
        \includegraphics[width=1.0\textwidth]{../output/plots/definition}
        \caption{definition}
        \label{fig:definition}
\end{figure}

\lipsum[6]

\begin{figure}[H]
    \centering
        \includegraphics[width=1.0\textwidth]{../output/plots/crowding_map}
        \caption{Crowding map}
        \label{fig:crowding_map}
\end{figure}

\lipsum[7]

\begin{figure}[H]
    \centering
        \includegraphics[width=1.0\textwidth]{../output/plots/country_crowding}
        \caption{countrycrowding}
        \label{fig:country_crowding}
\end{figure}

\lipsum[8]

\begin{figure}[H]
    \centering
        \includegraphics[width=1.0\textwidth]{../output/plots/interventions}
        \caption{Interventions}
        \label{fig:interventions}
\end{figure}

\subsection{Demand planning}
\lipsum[4]

\begin{figure}[H]
    \centering
    \begin{subfigure}[b]{0.4\textwidth}
        \includegraphics[width=\textwidth]{../output/plots/benefit}
        \caption{Would sufficiently accurate patient volume forecasting or crowding early warning software help alleviate the crowding problem?}
        \label{fig:benefit}
    \end{subfigure}
    \begin{subfigure}[b]{0.4\textwidth}
        \includegraphics[width=\textwidth]{../output/plots/usage}
        \caption{Do you currently use a patient volume forecasting or crowding early warning software to guide decision making?}
        \label{fig:usage}
    \end{subfigure}
    \caption{}
    \label{fig:usage}
\end{figure}

\lipsum[5]

\begin{figure}[H]
    \centering
        \includegraphics[width=1.0\textwidth]{../output/plots/horizon}
        \caption{Horizon}
        \label{fig:horizon}
\end{figure}

\lipsum[6]

\begin{figure}[H]
    \centering
        \includegraphics[width=1.0\textwidth]{../output/plots/enabled}
        \caption{Enabled}
        \label{fig:enabled}
\end{figure}

\section{Discussion}
\lipsum[1-8]

\end{document}
