\documentclass{article}
\usepackage{booktabs}
\usepackage{graphicx}
\usepackage{lipsum}
\usepackage{float}
\usepackage{subcaption}
\usepackage{multirow}
\usepackage{rotating}
\usepackage{makecell}
\usepackage{xcolor}
\usepackage{amsmath}


\title{Emergency department crowding in Europe: Causes, effects, extent and solutions as evaluated by 364 emergency department physicians}
\author{European emergency department crowding study consortium\footnote{EEDCS was conducted as a joint effort by the Research commitee of the European Society for Emergency Medicine (EUSEM), EUSEM Research Network (EUSEM RN) and Tampere Univeristy, Finland. The full list of authors along their contributions is provided at the end of the article.}}

\begin{document}
\maketitle


\begin{abstract}
    \lipsum[1]    
\end{abstract}

\section{Introduction}

Healthcare systems worldwide are facing an unprecedented challenge. The dependency ratio, which describes the ratio between of people dependent of other people, has reached x by the time of writing this manuscript and is predicted to reach the level of y by 2080. It means that by 2080, there will be d dependent persons for each p providing person. These are unprecedented numbers: the highest dependency ratio in OECD countries ever recorded is z by 1970. The problem is getting worse year by year, but the results of the problem are evident already today. 

One of the most prominent manifestations of this problem is emergency department crowding. Over the last x years, crowding has attracted an increasing amount of attention in both academia and media a like and the problem. Some collagues such as z have even suggested that crowding has reached the level of publich healthcare crisis. The amount of attention to this problem is warranted since the implications of the crowding problem has been extensively documented. For example, the association between 10- or 30-day mortality and crowding has been repeatedly demonstrated at least in Sweden, United Kingdom, South Korea, Australia and United States. However, contrary to many other public healthcare issues, crowding has captured the imagination of popular media as well as depicted in the TV series the Pitt. Regardless of this increasing attention in academia, media and popular culture the problem only seems to get worse. We believe this is in part due to several gaps in current literature.

First, although crowding receives an increasing amount attention, it is still unclear what clinical practitioners and ED management means by crowding. Over the years, frameworks such as NEDOCS or EDWIN have been proposed but the extent to which they have been utilized remain unclear. Moreover, it is unclear how different countries are affected by the problem and if one country suffers from the issue more than another. This itself could help to understand the nature of crowding and point a way to possible solutions.

Second, the causes of crowding have been hypothesised and it is certainly easy to come up with explanations based on ones clinical experience. However, the opinions of the clinicians working in the European ED's has not been systematically investigated. This is a problem because solving the crowding problem is impossible without identifying the root causes and targeting those with appropriate interventions.

Third, there is an increasing interest in applying demand management solutions to the crowding problem. The rationale here is simple and well-justified: if we can predict when the ED will be crowded, we can enable pre-emptive meausures to prevent crowding and save lives in the process. This is analogous to patient monitoring that is extensively done in all hospitals dedicated to quality care: instead of utilizing a national early warning score to predict the probability of death for a single patient, ED demand management effort aims to create an early warning score to predict the probability of death for a cohort of ED patients. These articles have reached a level of high mathematical and compuational complexity and demonstrated sometimes impressive performance retrospectively. However, we have seen very little engagement with the ED stakeholders regarding many important questions in this area such as: do clinicians actually want a predictive model to their ED? If yes, would it enable relevant interventions to alleviate crowding problem? Who -- if anyone -- is able to act based on the information? What should we actually predict for: arrivals, occupancy or something else?

In this article, we aim to bridge the gap between policymakers, emergency department physicians and mathematicians by asking these unclear questions from the people who know the answers first hand: those clinicians who work in the front lines of the emergency departments all over Europe.

% This might seem mundane but it demonstrates something important: almost everyone has a first hand experience visiting an ED and that experience ends up shaping the image of the healthcare system.

  
\textcolor{red}{Lisäanalyysi: ruuhka terveydenhuollon kustannusten funktiona}
\textcolor{red}{Lisäanalyysi: ruuhka yksityinen vs. julkinen terveydenhuolto}



\section{Materials and methods}
\lipsum[9-11]

Respondent level: Unprocessed respondent data

Hospital-level mean

\begin{equation}
\bar{y}_{jk} = \frac{1}{n_{jk}} \sum_{i=1}^{n_{jk}} y_{ijk}
\end{equation}

where $y_{ijk}$ is response of respondent $i$ in hospital j in country k,
and $n_{jk}$ is number of respondents in that hospital.

Country-level mean (mean of hospital means)

\begin{equation}
\bar{y}{k}^{(H)} = \frac{1}{m_k} \sum{j=1}^{m_k} \bar{y}_{jk}
\end{equation}

where $m_k = number of hospitals in country k$.

(If you prefer the respondent-weighted version instead, use)

\begin{equation}
\bar{y}{k}^{(R)} = \frac{1}{N_k} \sum{j=1}^{m_k} n_{jk} \, \bar{y}_{jk}
\end{equation}

with \[N_k = \sum_{j=1}^{m_k} n_{jk}.\]

Europe-level population-weighted mean

\begin{equation}
\bar{y}{E} = \frac{\sum{k=1}^{K} w_k \, \bar{y}{k}^{(H)}}{\sum{k=1}^{K} w_k}
\end{equation}

where $w_k$ is population weight (or sample weight) for country k,
and K is number of countries in Europe.



\section{Results}


\subsection{Respondents}
\lipsum[4-6]
% \begin{table}[h]
\caption{Repondent positions}
\label{tab:respondents}
\begin{tabular}{lrrrrr}
\toprule
 & Head & Other & Resident & Specialist & Trainee \\
country &  &  &  &  &  \\
\midrule
Albania & 3 & 0 & 0 & 4 & 0 \\
Austria & 1 & 0 & 0 & 3 & 0 \\
Belgium & 1 & 0 & 0 & 0 & 0 \\
Croatia & 4 & 0 & 4 & 8 & 0 \\
Denmark & 1 & 0 & 3 & 10 & 0 \\
Estonia & 3 & 1 & 0 & 5 & 0 \\
Finland & 7 & 1 & 3 & 3 & 0 \\
France & 3 & 0 & 1 & 11 & 0 \\
Germany & 20 & 3 & 0 & 5 & 1 \\
Greece & 4 & 0 & 3 & 5 & 0 \\
Hungary & 7 & 0 & 1 & 3 & 0 \\
Iceland & 1 & 0 & 0 & 4 & 0 \\
Ireland & 1 & 0 & 1 & 5 & 1 \\
Italy & 8 & 3 & 0 & 9 & 0 \\
Lithuania & 0 & 0 & 0 & 1 & 0 \\
Malta & 0 & 0 & 0 & 7 & 1 \\
Netherlands & 0 & 0 & 0 & 6 & 0 \\
Norway & 3 & 0 & 0 & 1 & 0 \\
Poland & 8 & 1 & 3 & 4 & 0 \\
Romania & 3 & 0 & 0 & 2 & 0 \\
Slovenia & 2 & 0 & 1 & 1 & 0 \\
Spain & 3 & 1 & 0 & 9 & 0 \\
Sweden & 2 & 1 & 0 & 7 & 0 \\
Switzerland & 1 & 0 & 0 & 0 & 0 \\
Turkey & 0 & 6 & 16 & 14 & 3 \\
United Kingdom & 5 & 14 & 29 & 36 & 8 \\
\bottomrule
\end{tabular}
\end{table}

% \begin{table}[h]
\caption{Hospital types}
\label{tab:hospitals}
\begin{tabular}{lrrrrr}
\toprule
 & other & primary & secondary & specialized & tertiary \\
country &  &  &  &  &  \\
\midrule
Albania & 0 & 1 & 3 & 0 & 3 \\
Austria & 0 & 0 & 0 & 0 & 4 \\
Belgium & 0 & 0 & 0 & 0 & 1 \\
Croatia & 0 & 2 & 1 & 0 & 13 \\
Denmark & 0 & 0 & 8 & 0 & 6 \\
Estonia & 0 & 2 & 2 & 0 & 5 \\
Finland & 0 & 0 & 8 & 0 & 6 \\
France & 0 & 3 & 1 & 1 & 10 \\
Germany & 0 & 0 & 8 & 0 & 21 \\
Greece & 0 & 0 & 2 & 0 & 10 \\
Hungary & 1 & 0 & 5 & 0 & 5 \\
Iceland & 0 & 1 & 0 & 0 & 4 \\
Ireland & 1 & 0 & 2 & 0 & 5 \\
Italy & 0 & 3 & 4 & 2 & 11 \\
Lithuania & 0 & 0 & 0 & 0 & 1 \\
Malta & 0 & 1 & 1 & 0 & 6 \\
Netherlands & 0 & 0 & 4 & 0 & 2 \\
Norway & 0 & 0 & 3 & 0 & 1 \\
Poland & 1 & 0 & 6 & 0 & 9 \\
Romania & 0 & 0 & 1 & 0 & 4 \\
Slovenia & 0 & 0 & 1 & 0 & 3 \\
Spain & 0 & 3 & 2 & 1 & 7 \\
Sweden & 1 & 2 & 2 & 0 & 5 \\
Switzerland & 0 & 0 & 0 & 0 & 1 \\
Turkey & 1 & 0 & 4 & 2 & 32 \\
United Kingdom & 2 & 5 & 49 & 3 & 33 \\
\bottomrule
\end{tabular}
\end{table}

\begin{table}[H]
\caption{Description}
\label{tab:description}
\footnotesize
\makebox[\textwidth][c]{%
\resizebox{1.2\textwidth}{!}{%
\begin{tabular}{p{1.0cm}p{1.1cm}p{0.07cm}p{0.07cm}p{0.07cm}p{0.07cm}p{0.07cm}p{0.07cm}p{0.07cm}p{0.07cm}p{0.07cm}p{0.07cm}p{0.07cm}p{0.07cm}p{0.07cm}p{0.07cm}p{0.07cm}p{0.07cm}p{0.07cm}p{0.07cm}p{0.07cm}p{0.07cm}p{0.07cm}p{0.07cm}p{0.07cm}p{0.07cm}p{0.07cm}p{0.07cm}p{0.07cm}}
\toprule
 &  & \rotatebox{90}{Albania} & \rotatebox{90}{Austria} & \rotatebox{90}{Belgium} & \rotatebox{90}{Croatia} & \rotatebox{90}{Denmark} & \rotatebox{90}{Estonia} & \rotatebox{90}{Finland} & \rotatebox{90}{France} & \rotatebox{90}{Germany} & \rotatebox{90}{Greece} & \rotatebox{90}{Hungary} & \rotatebox{90}{Iceland} & \rotatebox{90}{Ireland} & \rotatebox{90}{Italy} & \rotatebox{90}{Lithuania} & \rotatebox{90}{Malta} & \rotatebox{90}{Netherlands} & \rotatebox{90}{Norway} & \rotatebox{90}{Poland} & \rotatebox{90}{Romania} & \rotatebox{90}{Slovenia} & \rotatebox{90}{Spain} & \rotatebox{90}{Sweden} & \rotatebox{90}{Switzerland} & \rotatebox{90}{Turkey} & \rotatebox{90}{United Kingdom} & \rotatebox{90}{Total} \\
\midrule
\multirow[t]{5}{*}{Position} & Other & 0 & 0 & 0 & 0 & 0 & 1 & 1 & 0 & 3 & 0 & 0 & 0 & 0 & 3 & 0 & 0 & 0 & 0 & 1 & 0 & 0 & 1 & 1 & 0 & 6 & 14 & 31 \\
 & Trainee & 0 & 0 & 0 & 0 & 0 & 0 & 0 & 0 & 1 & 0 & 0 & 0 & 1 & 0 & 0 & 1 & 0 & 0 & 0 & 0 & 0 & 0 & 0 & 0 & 3 & 8 & 14 \\
 & Resident & 0 & 0 & 0 & 4 & 3 & 0 & 3 & 1 & 0 & 3 & 1 & 0 & 1 & 0 & 0 & 0 & 0 & 0 & 3 & 0 & 1 & 0 & 0 & 0 & 16 & 29 & 65 \\
 & Specialist & 4 & 3 & 0 & 8 & 10 & 5 & 3 & 11 & 5 & 5 & 3 & 4 & 5 & 9 & 1 & 7 & 6 & 1 & 4 & 2 & 1 & 9 & 7 & 0 & 14 & 36 & 163 \\
 & Head & 3 & 1 & 1 & 4 & 1 & 3 & 7 & 3 & 20 & 4 & 7 & 1 & 1 & 8 & 0 & 0 & 0 & 3 & 8 & 3 & 2 & 3 & 2 & 1 & 0 & 5 & 91 \\
\cline{1-29}
\multirow[t]{4}{*}{Hospital} & Other & 0 & 0 & 0 & 0 & 0 & 0 & 0 & 0 & 0 & 0 & 1 & 0 & 1 & 0 & 0 & 0 & 0 & 0 & 1 & 0 & 0 & 0 & 1 & 0 & 1 & 2 & 7 \\
 & Primary & 1 & 0 & 0 & 2 & 0 & 2 & 0 & 3 & 0 & 0 & 0 & 1 & 0 & 3 & 0 & 1 & 0 & 0 & 0 & 0 & 0 & 3 & 2 & 0 & 0 & 5 & 23 \\
 & Secondary & 3 & 0 & 0 & 1 & 8 & 2 & 8 & 1 & 8 & 2 & 5 & 0 & 2 & 4 & 0 & 1 & 4 & 3 & 6 & 1 & 1 & 2 & 2 & 0 & 4 & 49 & 117 \\
 & Tertiary & 3 & 4 & 1 & 13 & 6 & 5 & 6 & 11 & 21 & 10 & 5 & 4 & 5 & 13 & 1 & 6 & 2 & 1 & 9 & 4 & 3 & 8 & 5 & 1 & 34 & 36 & 217 \\
\cline{1-29}
 & Total & 7 & 4 & 1 & 16 & 14 & 9 & 14 & 15 & 29 & 12 & 11 & 5 & 8 & 20 & 1 & 8 & 6 & 4 & 16 & 5 & 4 & 13 & 10 & 1 & 39 & 92 & 364 \\
\cline{1-29}
\bottomrule
\end{tabular}
}}
\end{table}


% \begin{figure}[H]
%     \centering
%         \includegraphics[width=0.7\textwidth]{../output/plots/country_freqs}
%         \caption{countryfreqs}
%         \label{fig:country_freqs}
% \end{figure}

\subsection{Causes}
\lipsum[4-6]

\begin{figure}[p]
    \centering
        \includegraphics[height=1.0\textheight]{../output/plots/causes}
        \caption{Perceived causes of crowding based on EU-level statistics.}
        \label{fig:causes}
\end{figure}

\subsection{Operations management}
\lipsum[4]

\begin{figure}[H]
    \centering
        \includegraphics[width=1.0\textwidth]{../output/plots/severity}
        \caption{Perceived severity of emergency department crowding based on country level statistics}
        \label{fig:severity}
\end{figure}

\lipsum[5]

\begin{figure}[H]
    \centering
        \includegraphics[width=1.0\textwidth]{../output/plots/definition}
        \caption{Distribution of utilized crowding definitions aggregated to EU level}
        \label{fig:definition}
\end{figure}

\lipsum[6]

\begin{figure}[H]
    \centering
        \includegraphics[width=1.0\textwidth]{../output/plots/crowding_map}
        \caption{Map of the proportion of crowded days during a normal operating month as estimated by the respondents and aggregated to country level.}
        \label{fig:crowding_map}
\end{figure}

\lipsum[7]

\begin{figure}[H]
    \centering
        \includegraphics[width=1.0\textwidth]{../output/plots/country_crowding}
        \caption{The proportion of crowded days during a normal operating month as estimated by the respondents.}
        \label{fig:country_crowding}
\end{figure}

\lipsum[8]

\begin{figure}[H]
    \centering
        \includegraphics[width=1.0\textwidth]{../output/plots/interventions}
        \caption{Interventions}
        \label{fig:interventions}
\end{figure}


\subsection{Demand planning}
\lipsum[4]

\begin{figure}[H]
    \centering
    \begin{subfigure}[b]{0.4\textwidth}
        \includegraphics[width=\textwidth]{../output/plots/benefit}
        \caption{Would sufficiently accurate patient volume forecasting or crowding early warning software help alleviate the crowding problem?}
        \label{fig:benefit}
    \end{subfigure}
    \begin{subfigure}[b]{0.4\textwidth}
        \includegraphics[width=\textwidth]{../output/plots/usage}
        \caption{Do you currently use a patient volume forecasting or crowding early warning software to guide decision making?}
        \label{fig:usage}
    \end{subfigure}
    \caption{}
    \label{fig:usage}
\end{figure}

\lipsum[5]

\begin{figure}[H]
    \centering
        \includegraphics[width=1.0\textwidth]{../output/plots/horizon}
        \caption{Horizon}
        \label{fig:horizon}
\end{figure}

\lipsum[6]

\begin{figure}[H]
    \centering
        \includegraphics[width=0.5\textwidth]{../output/plots/enabled}
        \caption{Enabled}
        \label{fig:enabled}
\end{figure}

\section{Discussion}
\lipsum[1-8]

\end{document}
