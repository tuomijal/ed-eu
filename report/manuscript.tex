\documentclass{article}
\usepackage{booktabs}
\usepackage{graphicx}
\usepackage{lipsum}
\usepackage{float}
\usepackage{subcaption}
\usepackage{multirow}
\usepackage{rotating}
\usepackage{makecell}
\usepackage{xcolor}
\usepackage{amsmath}


\title{Emergency department crowding in Europe: Causes, effects, extent and solutions as evaluated by 364 emergency department physicians from 26 European countries}

\author{European emergency department crowding study consortium\footnote{EEDCS was conducted as a joint effort by the Research commitee of the European Society for Emergency Medicine (EUSEM RC), EUSEM Research Network (EUSEM RN) and Tampere Univeristy, Finland. The full list of authors along their contributions is provided at the end of the article.}}

\begin{document}
\maketitle


\begin{abstract}
    Healthcare systems worldwide are struggling with the burden of ageing populations and with the ever increasing demand for healthcare services it creates. One prominent manifestation of this issue is emergency department crowding which has been established as a severe public health issue that severely compromises patient safety all around the world. In this extensive survey, we report the perspective of 364 clinicians from 25 European countries on the causes, extent, effects and solutions to the crowding problem.
\end{abstract}

\section{Introduction}

Healthcare systems worldwide are facing an unprecedented challenge. The dependency ratio, which describes the ratio between of people dependent of other people, has reached x by the time of writing this manuscript and is predicted to reach the level of y by 2080. It means that by 2080, there will be d dependent persons for each p providing person. These are unprecedented numbers: the highest dependency ratio in OECD countries ever recorded is z by 1970. The problem is getting worse year by year, but the results of the problem are evident already today. 

One of the most prominent manifestations of this problem is emergency department crowding. Over the last x years, crowding has attracted an increasing amount of attention in both academia and media a like and the problem. Some collagues such as z have even suggested that crowding has reached the level of publich healthcare crisis. The amount of attention to this problem is warranted since the implications of the crowding problem has been extensively documented. For example, the association between 10- or 30-day mortality and crowding has been repeatedly demonstrated at least in Sweden, United Kingdom, South Korea, Australia and United States. However, contrary to many other public healthcare issues, crowding has captured the imagination of popular media as well as depicted in the TV series the Pitt. Regardless of this increasing attention in academia, media and popular culture the problem only seems to get worse. We believe this is in part due to several gaps in current literature.

First, although crowding receives an increasing amount attention, it is still unclear what clinical practitioners and ED management means by crowding. Over the years, frameworks such as NEDOCS or EDWIN have been proposed but the extent to which they have been utilized remain unclear. Moreover, it is unclear how different countries are affected by the problem and if one country suffers from the issue more than another. This itself could help to understand the nature of crowding and point a way to possible solutions.

Second, the causes of crowding have been hypothesised and it is certainly easy to come up with explanations based on ones clinical experience. However, the opinions of the clinicians working in the European ED's has not been systematically investigated. This is a problem because solving the crowding problem is impossible without identifying the root causes and targeting those with appropriate interventions.

Third, there is an increasing interest in applying demand management solutions to the crowding problem. The rationale here is simple and well-justified: if we can predict when the ED will be crowded, we can enable pre-emptive meausures to prevent crowding and save lives in the process. This is analogous to patient monitoring that is extensively done in all hospitals dedicated to quality care: instead of utilizing a national early warning score to predict the probability of death for a single patient, ED demand management effort aims to create an early warning score to predict the probability of death for a cohort of ED patients. These articles have reached a level of high mathematical and compuational complexity and demonstrated sometimes impressive performance retrospectively. However, we have seen very little engagement with the ED stakeholders regarding many important questions in this area such as: do clinicians actually want a predictive model to their ED? If yes, would it enable relevant interventions to alleviate crowding problem? Who -- if anyone -- is able to act based on the information? What should we actually predict for: arrivals, occupancy or something else?

In this article, we aim to bridge the gap between policymakers, emergency department physicians and mathematicians by asking these unclear questions from the people who know the answers first hand: those clinicians who work in the front lines of the emergency departments all over Europe.

% This might seem mundane but it demonstrates something important: almost everyone has a first hand experience visiting an ED and that experience ends up shaping the image of the healthcare system.

  
\textcolor{red}{Lisäanalyysi: ruuhka terveydenhuollon kustannusten funktiona}
\textcolor{red}{Lisäanalyysi: ruuhka yksityinen vs. julkinen terveydenhuolto}



\section{Materials and methods}

\subsection{Data collection}

Data were collected using a structured survey questionnaire administered through the REDCap (Research Electronic Data Capture) electronic data capture system. The survey was distributed by identifying country leads for each participating country through both EUSEM RC and EUSEM RN and each county lead was tasked to recruit x to y representative respondents. The data collection began on y of z, 2025 and ended on c of b, 2025. 

\subsection{Standardization}

The data collection method allowed multiple responses from the same hospital and multiple responses from each country but did not standardize the number of responses over hospitals and countries. This approach can result in an unbalanced dataset where high number of responses in one hospital may bias the results of the country and where high number of responses from one country can bias the results of the whole of Europe. For this reason, a set of standardizations were performed on each of these levels, which are described below.

\paragraph{Respondent level} refers to the initial dataset where each row represents a response from one pariticipant. This level answers the question: what do the respondents of the survey think of this specific question. For the sake of clarity, each response will be further denoted as follows:

\begin{equation}
	\texttt{response} = \texttt{r}
\end{equation}

\paragraph{Hospital-level} refers to the responses standardized to the level of each hospital, where each row represents the mean of the responses $\texttt{r}$ from each hospital $\texttt{h}$. This level of standardization removes the bias that would otherwise be caused by many responses from one hospital and depicts the consensus of a hospitals based on the multiple responses. It answers the quetions: what do the individual hospitals think of a specific question? For the sake of clarity, hospital level responses can be denoted as follows:

\begin{equation}
	\texttt{h}_i = \frac{1}{n_i}\sum_{j=1}^{n_i} \texttt{r}_{ij}
\end{equation}

where $\texttt{h}_i$ is the hospital-level response for hospital $i$, $\texttt{r}_{ij}$ is the $j$-th response from hospital $i$, and $n_i$ is the number of responses from hospital $i$.


\paragraph{Country-level} means the hospital-level responses standardized to the level of each county, where each row represents the mean of the hospital level responses $\texttt{h}$ of each country. This level of standardizatino removes the bias potentially caused by disproportionate number of participating hospitals in one country. It answers the question: what does the country as a whole think of a specific question? Mathematically, country level responses take the following form:

\begin{equation}
	\texttt{c}_k = \frac{1}{m_k}\sum_{i=1}^{m_k} \texttt{h}_{ki}
\end{equation}

where $\texttt{c}_k$ is the country-level response for country $k$, $\texttt{h}_{ki}$ is the hospital-level response for the $i$-th hospital in country $k$, and $m_k$ is the number of hospitals in country $k$.





\paragraph{Europe-level} is finally the consensus of the whole of Europe based on the three aforementioned levels. It is a population-weighted mean of the country level responses and it removes the potential bias of small countries having a disproportionate weight on the consenus of the whole of Europe in relation to countries with bigger population. The population of each responding country was extracted from \textcolor{red}{tarkista}. The EU level takes thus the following matheamatical form:

\begin{equation}
	\texttt{EU} = \sum_{k=1}^{K} w_k \cdot \texttt{c}_k
\end{equation}

where $\texttt{EU}$ is the Europe-level response, $\texttt{c}_k$ is the country-level response for country $k$, $w_k = \frac{p_k}{\sum_{k=1}^{K} p_k}$ is the population weight for country $k$, $p_k$ is the population of country $k$, and $K$ is the total number of countries.

Later, whenever results are represented or discussed, the specific level is expressed using the terminology above.

\section{Results}


\subsection{Respondents}

The respondent characteristics are presented in Table \ref{tab:description}. There were a total of 364 responses from 254 individual hospitals and from 26 individual countries. The number of responses from different countries was heavily Pareto distributed: for example the United Kindgom and Turkey provided 92 and 39 responses respectively whereas some countires such as Belgium, Lithuania or Switzerland had only one respondent each. 163 (45\%) of respondents were specialists, 91 (25\%) department heads, 65 (18\%) residents and 14 (4\%) were trainees. The remaining 31 (9\%) respondents described their position as \emph{Other}. Most of the responses were provided by physicians working at a tertiary hospital 217 (59\%) followed by secondary care with 117 (32\%) and primary care hospitals with 23 (6\%) responses. 7 (2\%) reported their hospital to belong to \emph{Other} category.


\begin{table}[H]
\caption{Description}
\label{tab:description}
\footnotesize
\makebox[\textwidth][c]{%
\resizebox{1.2\textwidth}{!}{%
\begin{tabular}{p{1.0cm}p{1.1cm}p{0.07cm}p{0.07cm}p{0.07cm}p{0.07cm}p{0.07cm}p{0.07cm}p{0.07cm}p{0.07cm}p{0.07cm}p{0.07cm}p{0.07cm}p{0.07cm}p{0.07cm}p{0.07cm}p{0.07cm}p{0.07cm}p{0.07cm}p{0.07cm}p{0.07cm}p{0.07cm}p{0.07cm}p{0.07cm}p{0.07cm}p{0.07cm}p{0.07cm}p{0.07cm}p{0.07cm}}
\toprule
 &  & \rotatebox{90}{Albania} & \rotatebox{90}{Austria} & \rotatebox{90}{Belgium} & \rotatebox{90}{Croatia} & \rotatebox{90}{Denmark} & \rotatebox{90}{Estonia} & \rotatebox{90}{Finland} & \rotatebox{90}{France} & \rotatebox{90}{Germany} & \rotatebox{90}{Greece} & \rotatebox{90}{Hungary} & \rotatebox{90}{Iceland} & \rotatebox{90}{Ireland} & \rotatebox{90}{Italy} & \rotatebox{90}{Lithuania} & \rotatebox{90}{Malta} & \rotatebox{90}{Netherlands} & \rotatebox{90}{Norway} & \rotatebox{90}{Poland} & \rotatebox{90}{Romania} & \rotatebox{90}{Slovenia} & \rotatebox{90}{Spain} & \rotatebox{90}{Sweden} & \rotatebox{90}{Switzerland} & \rotatebox{90}{Turkey} & \rotatebox{90}{United Kingdom} & \rotatebox{90}{Total} \\
\midrule
\multirow[t]{5}{*}{Position} & Other & 0 & 0 & 0 & 0 & 0 & 1 & 1 & 0 & 3 & 0 & 0 & 0 & 0 & 3 & 0 & 0 & 0 & 0 & 1 & 0 & 0 & 1 & 1 & 0 & 6 & 14 & 31 \\
 & Trainee & 0 & 0 & 0 & 0 & 0 & 0 & 0 & 0 & 1 & 0 & 0 & 0 & 1 & 0 & 0 & 1 & 0 & 0 & 0 & 0 & 0 & 0 & 0 & 0 & 3 & 8 & 14 \\
 & Resident & 0 & 0 & 0 & 4 & 3 & 0 & 3 & 1 & 0 & 3 & 1 & 0 & 1 & 0 & 0 & 0 & 0 & 0 & 3 & 0 & 1 & 0 & 0 & 0 & 16 & 29 & 65 \\
 & Specialist & 4 & 3 & 0 & 8 & 10 & 5 & 3 & 11 & 5 & 5 & 3 & 4 & 5 & 9 & 1 & 7 & 6 & 1 & 4 & 2 & 1 & 9 & 7 & 0 & 14 & 36 & 163 \\
 & Head & 3 & 1 & 1 & 4 & 1 & 3 & 7 & 3 & 20 & 4 & 7 & 1 & 1 & 8 & 0 & 0 & 0 & 3 & 8 & 3 & 2 & 3 & 2 & 1 & 0 & 5 & 91 \\
\cline{1-29}
\multirow[t]{4}{*}{Hospital} & Other & 0 & 0 & 0 & 0 & 0 & 0 & 0 & 0 & 0 & 0 & 1 & 0 & 1 & 0 & 0 & 0 & 0 & 0 & 1 & 0 & 0 & 0 & 1 & 0 & 1 & 2 & 7 \\
 & Primary & 1 & 0 & 0 & 2 & 0 & 2 & 0 & 3 & 0 & 0 & 0 & 1 & 0 & 3 & 0 & 1 & 0 & 0 & 0 & 0 & 0 & 3 & 2 & 0 & 0 & 5 & 23 \\
 & Secondary & 3 & 0 & 0 & 1 & 8 & 2 & 8 & 1 & 8 & 2 & 5 & 0 & 2 & 4 & 0 & 1 & 4 & 3 & 6 & 1 & 1 & 2 & 2 & 0 & 4 & 49 & 117 \\
 & Tertiary & 3 & 4 & 1 & 13 & 6 & 5 & 6 & 11 & 21 & 10 & 5 & 4 & 5 & 13 & 1 & 6 & 2 & 1 & 9 & 4 & 3 & 8 & 5 & 1 & 34 & 36 & 217 \\
\cline{1-29}
 & Total & 7 & 4 & 1 & 16 & 14 & 9 & 14 & 15 & 29 & 12 & 11 & 5 & 8 & 20 & 1 & 8 & 6 & 4 & 16 & 5 & 4 & 13 & 10 & 1 & 39 & 92 & 364 \\
\cline{1-29}
\bottomrule
\end{tabular}
}}
\end{table}


\subsection{Extent, definitions and consequenses}
\lipsum[4]

\begin{figure}[H]
    \centering
        \includegraphics[width=1.0\textwidth]{../output/plots/crowding_map}
        \caption{Map of the proportion of crowded days during a normal operating month as estimated by the respondents and aggregated to country level.}
        \label{fig:crowding_map}
\end{figure}

\lipsum[7]

\begin{figure}[H]
    \centering
        \includegraphics[width=1.0\textwidth]{../output/plots/country_crowding}
        \caption{The proportion of crowded days during a normal operating month as estimated by the respondents.}
        \label{fig:country_crowding}
\end{figure}

\lipsum[5]

\begin{figure}[H]
    \centering
        \includegraphics[width=1.0\textwidth]{../output/plots/severity}
        \caption{Perceived severity of emergency department crowding based on country level statistics}
        \label{fig:severity}
\end{figure}


\begin{figure}[H]
    \centering
        \includegraphics[width=1.0\textwidth]{../output/plots/definition}
        \caption{Distribution of utilized crowding definitions aggregated to EU level}
        \label{fig:definition}
\end{figure}

\lipsum[6]

\subsection{Causes}
\lipsum[4-6]

\begin{figure}[p]
    \centering
        \includegraphics[height=1.0\textheight]{../output/plots/causes}
        \caption{Perceived causes of crowding based on EU-level statistics.}
        \label{fig:causes}
\end{figure}


\subsection{Interventions and demand planning}

\lipsum[8]

\begin{figure}[H]
    \centering
        \includegraphics[width=1.0\textwidth]{../output/plots/interventions}
        \caption{Interventions}
        \label{fig:interventions}
\end{figure}


\lipsum[4]

\begin{figure}[H]
    \centering
    \begin{subfigure}[b]{0.4\textwidth}
        \includegraphics[width=\textwidth]{../output/plots/benefit}
        \caption{Would sufficiently accurate patient volume forecasting or crowding early warning software help alleviate the crowding problem?}
        \label{fig:benefit}
    \end{subfigure}
    \begin{subfigure}[b]{0.4\textwidth}
        \includegraphics[width=\textwidth]{../output/plots/usage}
        \caption{Do you currently use a patient volume forecasting or crowding early warning software to guide decision making?}
        \label{fig:usage}
    \end{subfigure}
    \caption{}
    \label{fig:usage}
\end{figure}

\lipsum[5]

\begin{figure}[H]
    \centering
        \includegraphics[width=1.0\textwidth]{../output/plots/horizon}
        \caption{Horizon}
        \label{fig:horizon}
\end{figure}

\lipsum[6]

\begin{figure}[H]
    \centering
        \includegraphics[width=0.5\textwidth]{../output/plots/enabled}
        \caption{Enabled}
        \label{fig:enabled}
\end{figure}

\section{Discussion}


\paragraph{The extent, perceived implications and definitions}

There are three main findings regarding the extent and definition of crowding: i) crowding affects countries disproportionely, ii) crowding is poorly defined and iii) crowding is considered to be at least a moderate problem.



\paragraph{The causes stratified by Asplin's model}


\paragraph{Demand management and interventions}

\subsection{Limitations} The frequencies of the responses from each country followed a Pareto distribution where -- for example -- the United Kingdom provided a high number of responses (n=92) as compared to Belgium, Lithuania or Switzerland (n=1). Although the responses were standardized to eliminate the evident mathematical bias this creates, there is not a `post hoc` method to remove the 



\end{document}
